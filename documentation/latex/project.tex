\documentclass[10pt]{report}

\usepackage[utf8]{inputenc} % Required for inputting international characters
\usepackage[T1]{fontenc} % Output font encoding for international characters
\usepackage{graphicx} % images
\usepackage{fancyhdr} % headers and footers
\usepackage{parskip} % paragraph
\usepackage{geometry} % shapes
\usepackage{hyperref} % Links
\usepackage{pdflscape} % making a page landscape
% \usepackage{microtype} % used for formatting nicely and getting rid of hbox badness
% \usepackage{natbib}
\renewcommand{\bibname}{References}
\renewcommand{\baselinestretch}{1.5}

\bibliographystyle{unsrtnat}

% hyperlink setup
\hypersetup{
    colorlinks,
    citecolor=black,
    filecolor=black,
    linkcolor=black,
    urlcolor=black
}

% margins and page size
\geometry{
a4paper,
left=30mm,
top=25mm,
right=30mm,
bottom=25mm
}

\begin{document}

\begin{titlepage}
\center
{\huge\bfseries Report}  
\end{titlepage}
\tableofcontents
\chapter{Abstract}
This project aim is to create a Souls-Like game with an AI system that allows for in-fighting between creatures based on a food chain. I researched different AI techniques to pull this off and landed on GOAP - Goal Oriented Action Planning - as the system I would use in this game.
\chapter{Introduction}
\chapter{Output Summary}
\chapter{Literature Review}
My project focuses on creating a game in the souls-like genre where the enemies act in an eco-system where there is a food-chain hierarchy in which they can fight each other.

An intricate in-fighting system between multiple different AI agents is what I will be focusing on. One of the early examples of in-fighting was seen in the original Doom (1993)\cite{doom93}, where if an enemy-A got attacked by enemy-B, enemy-A will change it's target from player to enemy-B, this is a very simple system but offers many different gameplay opportunities, many levels were crafted with this in mind where the players would have to manipulate the AI by luring one enemy into another's line of fire.
This idea was later expanded by Ubisoft in their later Far Cry games where different clans can combat each other in random events, and animals can also combat each other, combining these together, there a re many random events where animals attack enemies that are already mid combat between you and another clan. In Far Cry Primal they expanded the animal system so that you could tame any animal in the world and make them your companion in gameplay.

This sort of systemic game design is created by making the different game systems aware of each other, and this inturn invites the player to use their creativity to make plans on how to execute their goals and creating unique experiences. This type of gameplay has been coined as emergent gameplay where complex systems \textit{emerge} from the interaction on relatively simple game mechanics.\cite{emergentGameplay}

There are different types of AI techniques that I could use, the following are the ones that I have looked into: Finite State Machines, Behaviour Trees, Goal Oriented Action Planning and Utility Based AI.

Finite state machines are the most rudimentary AI system, it is a system that I had implemented in my Advanced Games Tech Project, these are very easy and quick to implement for a simple AI system however it does not suit the complex AI behaviours I would like to build as when adding more complexity to a state machine, the "program flow becomes much more difficult to understand and creates a debugging nightmare"\cite{gameAiByExample}.

Behaviour Trees were popularised by Halo 2\cite{halo2} and are now the most common AI technique used in modern games, this is due to there being a streamlined flow/logic to the readability of the AI, this makes it easier to expand while keeping debugging simple. The design is built using nodes as modules, allowing nodes to be reused with minimal effort. There is also a 'blackboard' where shared knowledge is kept which the AI uses to make 'smarter' decisions while keeping memory usage to a minimum. This all happens while keeping a low computational overhead if implemented correctly. \cite{behaviourTrees}

Goal Oriented Action Planning (GOAP) was originally implemented in a game in F.E.A.R (First Encounter Assault Recon) by Jeff Orkin \cite{goap}. It is a system that is made up of many actions, these actions comprise of objects, preconditions and effects; an agent will decide on an action to execute based on their current situation and then work backwards through the preconditions to create an efficient plan of executing the actions. This is done using an A* path finding algorithm, the actions have different weightings based on how easy they are to execute from the current state of the agent. This system can be very easy to scale if the ground work is set up properly as all would have to be done is add more actions in. Similar to behaviour trees (BT), a blackboard is used to store shared data which the AI can use to make decisions. This system can have performance issues, as goals and plans will have to be calculated for every active agent very frequently, so increasing the number of agents in the scene will have a performance hit. This was an issue when F.E.A.R came out as the rats in the game used the same AI system as the enemies, and once a rat had spawn, it would not be destroyed, and this would accumulate over a long play session and cause performance issues. This system can create novel AI behaviour which isn't seen easily in BT and FSM AI design. This is the system I would like to implement as it would be easier to create AI that seems to behave 'organically' like they do in F.E.A.R.. \cite{goapTommyTompson} An example of this is given in Jeff Orkin's 2003 paper on the subject of applying GOAP\cite{applyingGoap}, he talks about a character X running out of ammunition in their weapon, given that their goal is still to eliminate the enemy, the planner will find another way of doing that - in the example, there is a laser that can be activated by both player and enemies - the planner will now find a combination of actions to execute in a particular order for the enemy to activate the laser to eliminate the player. This is a novel idea that the planner came up with by using what was in the environment, behaviour like this would be missed in games using behaviour trees as that particular course of action was not added into the tree.
%might not need to cite tommy tompson

\chapter{Methodology}
% methodology then tools used
\section{Software Design Methodology}

The AI system is based around GOAP which works really well as a modular system that can bring about novel ideas by agents.

\section{Tools}
%tools
% need to add
% - Audio Tool
% - UML tool
% - evaluation tool

\subsection{Unity (Game Engine)}
Unity was chosen as the game engine over Unreal Engine as I had some prior (albeit minimal) experience in using it; Unity used C\# for scripting which is, for me, a simpler language to write in over C++, even though I have used C++ more extensively over the past year in various projects. Unity also had better performance on the hardware and operating system I would be using for this project, while being smaller in file size.

As mentioned in my PDD, Unity has an extensive amount documentation available on their website which is easy to understand, they also have a forum site where there is a very active community in helping new-comers with any issues they face; as well as this, there are plenty of tutorials about Unity on their own website and on YouTube.

There is also an asset store built into Unity with many paid and free assets, ranging from scripts, models, materials, shaders, physics systems and more.

All of this combined, it was very difficult for me to not choose Unity as the Game Engine for my project.

\subsection{JetBrains Rider (IDE)}
JetBrains Rider was my choice of IDE as I have been using JetBrains products since the Programming in Java module in the first year, and I feel they provide a very good set of refactoring and debugging tools with a good amount of community made extensions for the program. Rider was created as a .NET IDE, and now has native Unity support, which is perfect for use with Unity.

My other option was Microsoft's Visual Studio, which I had used in the second and third year Games Tech modules. After using it, I did not like it much and had installed the ReSharper plugin by JetBrains, which brought some of the functionality from JetBrains IDEs. Given that JetBrains have their own IDE for Unity, I decided to go with that over Visual Studio.

\subsection{GitHub}
I used GitHub as my version control and backup system as I am very familiar with it. I have used it extensively over my time at the university.

\subsection{Blender}
I used Blender as my 3D program of choice as it is very intuitive to learn and very simple to use. It is also free and open source leading to very many tutorials for it along with great documentation. It has support for opening most 3D filetypes which is good for me as I will be getting assets from different locations; if there is a filetype that is not supported, then there is bound to be an extension for it, due to the many community mad extensions available.

\subsection{Adobe Photoshop}
This is my image editor of choice as I am very well versed in it and have been using it over the past 6 years and professionally for the past 2 years.

\subsection{}


\section{Milestone 1}

\subsection{References for implementation}
I did extensive reading through the FEAR SDK\cite{fearSDK} that contains the full AI source code to fully understand the implementation of GOAP.
I also read up on many other implementations of GOAP and all of them are done in a very similar way with only minor structural differences. 
Peter Klooster created a multi-threaded GOAP System for Unity3D which he originally used in his game Basher Beatdown \cite{basherGoap}. This is a pretty simple implementation that includes a visualiser for the system as well.
Anne from the YouTube channel TheHappieCat did a GOAP implementation in Unity \cite{happieGoapVideo} based on the papers by Jeff Orkin and an article by Brent Owens \cite{brentOwensGoap} (taking the core system from here). She made a very informative video on the topic as well as walking through the code with a live simple example.
On Jeff Orkin's website, there was a reference to ReGOAP by Luciano Ferraro, which is a very generic C\# GOAP library, that was used in Unity3D, but can be used anywhere. Compared to the other implementations, this has been officially released on the Unity Asset store, however the code seems to be quite verbose and is not well commented, giving me a hard time understanding how to use it.

Based on all this I started creating my own GOAP system, however, it became very time consuming to create, and based on the gannt chart I had made, I was running behind on schedule; as the main focus of the project is to create the interaction between these different AI agents, and not the core AI system itself, I decided to use the implementation provided by Brent Owens \cite{brentOwensGoapCode} which is under the MIT licence. This is a very bare bones implementation only consisting of 6 scripts, and with the video explanation by TheHappieCat this was the simplest one I could implement and still have a great amount of flexibility in modifying the code due to its loosely coupled nature.


\begin{thebibliography}{}
    \bibitem{doom93}
    id Software, id Software, (1993). DOOM. [computer game]. Available at: \url{https://store.bethesda.net/store/bethesda/en_IE/pd/productID.5361563100/currency.GBP}
    \bibitem{farcry}
    Ubisoft.com. 2021. Far Cry franchise. [online] Available at: \url{https://www.ubisoft.com/en-gb/franchise/far-cry/} [Accessed 9 February 2021].
    \bibitem{farCryPrimalCompanions}
    Thompson, T., 2018. Primal Instinct | Companion AI in Far Cry Primal. [online] Gamasutra.com. Available at: \url{https://gamasutra.com/blogs/TommyThompson/20180906/325967/Primal_Instinct__Companion_AI_in_Far_Cry_Primal.php} [Accessed 12 February 2021].
    \bibitem{emergentGameplay}
    "Le Gameplay emergent (in French)". jeuxvideo.com. 2006-01-19. Available at: \url{https://www.jeuxvideo.com/dossiers/00006203/le-gameplay-emergent.htm} [Accessed 9th February 2021]
    \bibitem{gameAiByExample}
    Buckland, M., 2010. Programming Game AI by Example. Burlington: Jones \& Bartlett Learning, LLC, p.46.
    \bibitem{halo2}
    Bungie, Microsoft Game Studios. Halo 2 [computer game]. 04/09/2004.
    \bibitem{behaviourTrees}
    Simpson, C., 2014. Behavior trees for AI: How they work. [online] Gamasutra.com. Available at: \url{https://www.gamasutra.com/blogs/ChrisSimpson/20140717/221339/Behavior_trees_for_AI_How_they_work.php} [Accessed 12 February 2021].
    \bibitem{goap}
    Orkin, J., n.d. Goal-Oriented Action Planning (GOAP). [online] Alumni.media.mit.edu. Available at: \url{http://alumni.media.mit.edu/~jorkin/goap.html} [Accessed 12 February 2021].
    \bibitem{applyingGoap}
    Orkin, J., n.d. Applying Goal-Oriented Action Planning to Games. [online] Alumni.media.mit.edu. Available at: \url{http://alumni.media.mit.edu/~jorkin/GOAP_draft_AIWisdom2_2003.pdf} [Accessed 12 February 2021].
    \bibitem{goapTommyTompson}
    Thompson, T., 2020. Building the AI of F.E.A.R. with Goal Oriented Action Planning. [online] Gamasutra.com. Available at: \url{https://www.gamasutra.com/blogs/TommyThompson/20200507/362417/Building_the_AI_of_FEAR_with_Goal_Oriented_Action_Planning.php} [Accessed 13 February 2021].
    \bibitem{fearSDK}
    Monolith Productions, Inc., 2004, Fear SDK 1.08 [code] Available at: \url{http://fear.filefront.com/file/FEAR_v108_SDK;71433} and \url{https://github.com/xfw5/Fear-SDK-1.08.git}
    \bibitem{basherGoap}
    Klooster, Peter, GOAP, a multi-threaded library for Unity [code] Available at: \url{https://github.com/crashkonijn/GOAP}, under Apache-2.0 licence
    \bibitem{happieGoapVideo}
    Anne, 2016. Combat AI for Action-Adventure Games Tutorial [Unity/C\#] [GOAP]. [video] Available at: \url{https://youtu.be/n6vn7d5R_2c} [Accessed 13 February 2021].
    \bibitem{brentOwensGoap}
    Owens, B., 2014. Goal Oriented Action Planning for a Smarter AI. [online] Game Development Envato Tuts+. Available at: \url{https://gamedevelopment.tutsplus.com/tutorials/goal-oriented-action-planning-for-a-smarter-ai--cms-20793} [Accessed 13 February 2021].
    \bibitem{brentOwensGoapCode}
    Owens, B., 2014. Goal Oriented Action Planning for a Smarter AI. [code]. Available at: \url{https://github.com/sploreg/goap} [Accessed 13 February 2021].


\end{thebibliography}
\end{document}